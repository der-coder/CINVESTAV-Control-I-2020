\item [3.20] Show that an $n \times n$
matrix has the property $\mathbf{A}^k = \mathbf{0}$ for
$k \geq m$ if and only if $\mathbf{A}$
has eigenvalues 0 with multiplicity $n$ and index $m$ or less.
Such a matrix is a \emph{nilpotent} matrix.

Assume that $\mathbb{A}$ is in Jordan form, such that
$\mathbf{A}^k = \mathbf{Q} \hat{\mathbf{A}}^k \mathbf{Q}^{-1} $.

Given the premise $\mathbf{A}^k = \mathbf{0}$ then
$\hat{\mathbf{A}}^k = \mathbf{0}$.
Recall that

\begin{equation*}
 \hat{\mathbf{A}} = \begin{bmatrix}
                     \lambda_1 & 0  & \dots & 0\\
                     0 & \lambda_2  & \dots & 0\\
                     \vdots & \vdots& \ddots & \vdots\\
                     0 & 0 & \dots &  \lambda_n
                    \end{bmatrix}
\end{equation*}

where $\lambda_i$ are the eigenvalues of $\mathbf{A}$.

For $\hat{\mathbf{A}}^k = \mathbf{0}$ to be true,
then the eigenvalues of $\mathbf{A}$ must be zero or there must be
an eigenvector $\lambda_j = 0$ with multiplicity $n$.

Let A be a diagonal matrix with Jordan blocks $\mathbf{A}_i$
with eigenvalues zero.
If $\mathbf{A}^k = \mathbf{0}$ then
$\mathbf{A}^k = \text{diag}\{\mathbf{A}_1, \mathbf{A}_2, \dots, \mathbf{A}_n\}= \mathbf{0}$.
For this to be valid for $k \geq m$, the order $n_i$ of all Jordan blocks
must be $n_i \leq m$.
Thus, the index of $\mathbf{A}$ must also be m or less.
