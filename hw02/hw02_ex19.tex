\item[3.19] Show that if $\lambda$ is an eigenvalue of
$\mathbf{A}$ with eigenvector $\mathbf{x}$,
then $f(\lambda)$ is an eigenvalue of
$f(\mathbf{A})$ with the same eigenvector $\mathbf{x}$.

We prove this statement by using the Cayley-Hamilton theorem.
If $\lambda$ is an eigenvalue of $\mathbf{A}$ then

\begin{equation*}
 \mathbf{A} \mathbf{x} = \lambda \mathbf{x} \rightarrow
 \mathbf{A}^n \mathbf{x} = \lambda^n \mathbf{x}
\end{equation*}

Then $f(\mathbf{A})$ can be expressed as a linear combination of
linear matrices.

\begin{align*}
 f(\mathbf{A}) &= \beta_0 \mathbf{I} + \beta_1 \mathbf{A} + \cdots + \beta_n \mathbf{A}^{n-1}\\
 &= \beta_0 + \beta_1 \lambda + \cdots + \beta_n \lambda^{n-1}\\
 f(\mathbf{A}) \mathbf{x} &= (\beta_0 + \beta_1 \lambda + \cdots + \beta_n \lambda^{n-1}) \mathbf{x}\\
 &= f(\lambda) \mathbf{x}
\end{align*}

This proves that $f(\lambda)$ is an eigenvalue of $f(\mathbf{A})$.
