\item [3.13] Find Jordan-form representations of the following matrices:

\begin{align*}
 \mathbf{A}_1 = \begin{bmatrix}
      1 & 4 & 10 \\
      0 & 2 & 0 \\
      0 & 0 & 3
     \end{bmatrix}
& \quad
\mathbf{A}_2 = \begin{bmatrix}
       0 & 1 & 0 \\
       0 & 0 & 1 \\
       -2 & -4 & -3
      \end{bmatrix}
\\
\mathbf{A}_3 = \begin{bmatrix}
       1 & 0 & -1 \\
       0 & 1 & 0 \\
       0 & 0 & 2
      \end{bmatrix}
& \quad
\mathbf{A}_4 = \begin{bmatrix}
       0 & 4 & 3\\
       0 & 20 & 16\\
       0 & -25 & -20
      \end{bmatrix}
\end{align*}

Note that all except $A_4$ can be diagonalized.

For all matrices, the Jordan form is obtained from the eigenvalues
and eigenvectors of the matrix.
This done by obtaining solutions to the following equations

\begin{align*}
\Delta (\lambda) &= \det(\lambda \mathbf{I} - \mathbf{A}) = 0\\
(\mathbf{A} - \lambda_i \mathbf{I}) \mathbf{q}_i &= 0
\end{align*}

\begin{itemize}
 \item $\mathbf{A}_1$

 The matrix is triangular, thus its eigenvalues are the values
 of the diagonal.
 \begin{equation*}
  \hat{\mathbf{A}}_1 = \begin{bmatrix}
                        1 & 0 &0\\0 & 2 & 0\\0 & 0&3
                       \end{bmatrix}
 \end{equation*}

We obtain $\mathbf{Q} = \begin{bmatrix}
                         \mathbf{q}_1 &\mathbf{q}_2 &\mathbf{q}_3
                        \end{bmatrix}
$.
% q1
\begin{align*}
 (\mathbf{A}-\lambda_1) \mathbf{q}_1 &= \mathbf{0}\\
 \begin{bmatrix}
    0 & 4 & 10 \\
     0 & 1 & 0 \\
    0 & 0 & 2
    \end{bmatrix}\mathbf{q}_1 &= \mathbf{0}
\end{align*}

\begin{align*}
  \begin{bmatrix}
    0 & 4 & 10 \\
     0 & 1 & 0 \\
    0 & 0 & 2
    \end{bmatrix}
    \begin{bmatrix}
     1 \\ 0 \\ 0
    \end{bmatrix}
    &= \mathbf{0}
\end{align*}
% q2
\begin{align*}
 (\mathbf{A}-\lambda_2) \mathbf{q}_2 &= \mathbf{0}\\
 \begin{bmatrix}
    -1 & 4 & 10 \\
     0 & 0 & 0 \\
    0 & 0 & 1
    \end{bmatrix}\mathbf{q}_2 &= \mathbf{0}
\end{align*}

\begin{align*}
  \begin{bmatrix}
    -1 & 4 & 10 \\
     0 & 0 & 0 \\
    0 & 0 & 1
    \end{bmatrix}
    \begin{bmatrix}
     4 \\ 1 \\ 0
    \end{bmatrix}
    &= \mathbf{0}
\end{align*}
% q3
\begin{align*}
 (\mathbf{A}-\lambda_3) \mathbf{q}_3 &= \mathbf{0}\\
 \begin{bmatrix}
    -2 & 4 & 10 \\
     0 & -1 & 0 \\
    0 & 0 & 0
    \end{bmatrix}\mathbf{q}_3 &= \mathbf{0}
\end{align*}

\begin{align*}
  \begin{bmatrix}
    -2 & 4 & 10 \\
     0 & -1 & 0 \\
    0 & 0 & 0
    \end{bmatrix}
    \begin{bmatrix}
     5 \\ 0 \\ 1
    \end{bmatrix}
    &= \mathbf{0}
\end{align*}

Thus the Jordan form is

 \begin{align*}
 \hat{\mathbf{A}}_1 &= \begin{bmatrix}
                        1 & 0 &0\\0 & 2 & 0\\0 & 0&3
                       \end{bmatrix}\\
  \mathbf{Q} &= \begin{bmatrix}
   1 & 4 & 5\\
   0 & 1 & 0\\
   0 & 0 &1
  \end{bmatrix}
 \end{align*}

 %%%%%%%%%%%%%%%%%%%%%%%%%%%%%%%%%%%%%%%%%%%%%%%%%%%%%%%

 \item $\mathbf{A}_2$

 We obtain the eigenvalues of the matrix.

 \begin{align*}
  \det(\lambda \mathbf{I} - \mathbf{A}_2) &=
  \det(\begin{bmatrix}
       \lambda & -1 & 0 \\
       0 & \lambda & -1 \\
       2 & 4 & \lambda+3
      \end{bmatrix})=\mathbf{0}\\
      &= \lambda (\lambda^2 +3\lambda +4) -2\\
      &= \lambda^3 +3\lambda^2 + 4 \lambda -2\\
      &= (\lambda +1) (\lambda +1 + j) (\lambda +1 - j)
 \end{align*}


  \begin{equation*}
  \hat{\mathbf{A}}_2 = \begin{bmatrix}
                        -1 & 0 &0\\0 & -1-j & 0\\0 & 0&-1+j
                       \end{bmatrix}
 \end{equation*}

 Determine the eigenvectors of matrix $\mathbf{A}_2$.

\begin{align*}
 (\mathbf{A}-\lambda_1) \mathbf{q}_1 &= \mathbf{0}\\
 \begin{bmatrix}
    1 & 1 & 0 \\
     0 & 1 & 1 \\
    -2 & -4 & -2
    \end{bmatrix}\mathbf{q}_1 &= \mathbf{0}
\end{align*}

\begin{align*}
  \begin{bmatrix}
     1 & 1 & 0 \\
     0 & 1 & 1 \\
    -2 & -4 & -2
    \end{bmatrix}
    \begin{bmatrix}
     1 \\ -1 \\ 1
    \end{bmatrix}
    &= \mathbf{0}
\end{align*}
% q2
\begin{align*}
 (\mathbf{A}-\lambda_2) \mathbf{q}_2 &= \mathbf{0}\\
 \begin{bmatrix}
     1+j & 1 & 0 \\
     0 & 1+j & 1 \\
    -2 & -4 & -2+j
    \end{bmatrix}\mathbf{q}_2 &= \mathbf{0}
\end{align*}

\begin{align*}
  \begin{bmatrix}
     1+j & 1 & 0 \\
     0 & 1+j & 1 \\
    -2 & -4 & -2+j
    \end{bmatrix}
    \begin{bmatrix}
     -1 \\ 1+j \\ -2j
    \end{bmatrix}
    &= \mathbf{0}
\end{align*}
% q3
\begin{align*}
 (\mathbf{A}-\lambda_3) \mathbf{q}_3 &= \mathbf{0}\\
 \begin{bmatrix}
     1-j & 1 & 0 \\
     0 & 1-j & 1 \\
    -2 & -4 & -2-j
    \end{bmatrix}\mathbf{q}_3 &= \mathbf{0}
\end{align*}

\begin{align*}
  \begin{bmatrix}
     1-j & 1 & 0 \\
     0 & 1-j & 1 \\
    -2 & -4 & -2-j
    \end{bmatrix}
    \begin{bmatrix}
     -1 \\ 1-j \\ 2j
    \end{bmatrix}
    &= \mathbf{0}
\end{align*}

Thus the Jordan form is
 \begin{align*}
 \hat{\mathbf{A}}_2 &= \begin{bmatrix}
                        -1 & 0 &0\\0 & -1-j & 0\\0 & 0&-1+j
                       \end{bmatrix}\\
  \mathbf{Q} &= \begin{bmatrix}
   1 & -1 & -1\\
   -1 & 1+j & 1-j\\
   1 & -2j &2j
  \end{bmatrix}
 \end{align*}

 %%%%%%%%%%%%%%%%%%%%%%%%%%%%%%%%%%%

 \item $\mathbf{A}_3$

 The matrix is a triangular matrix, just like $\mathbf{A}_1$.

  \begin{equation*}
  \hat{\mathbf{A}}_3 = \begin{bmatrix}
                        1 & 0 &0\\0 & 1 & 0\\0 & 0&2
                       \end{bmatrix}
 \end{equation*}

 We obtain $\mathbf{Q}$ by finding the eigenvectors of the matrix.

 \begin{align*}
 (\mathbf{A}-\lambda_1) \mathbf{q}_1 &= \mathbf{0}\\
 \begin{bmatrix}
    0 & 0 & -1 \\
     0 & 0 & 0 \\
    0 & 0 & 1
    \end{bmatrix}\mathbf{q}_1 &= \mathbf{0}
\end{align*}

\begin{align*}
  \begin{bmatrix}
     0 & 0 & -1 \\
     0 & 0 & 0 \\
    0 & 0 & 1
    \end{bmatrix}
    \begin{bmatrix}
     1 \\ 0 \\ 0
    \end{bmatrix}
    &= \mathbf{0}
\end{align*}
% q2
\begin{align*}
 (\mathbf{A}-\lambda_2) \mathbf{q}_2 &= \mathbf{0}\\
 \begin{bmatrix}
    0 & 0 & -1 \\
     0 & 0 & 0 \\
    0 & 0 & 1
    \end{bmatrix}\mathbf{q}_2 &= \mathbf{0}
\end{align*}

\begin{align*}
  \begin{bmatrix}
     0 & 0 & -1 \\
     0 & 0 & 0 \\
    0 & 0 & 1
    \end{bmatrix}
    \begin{bmatrix}
     0 \\ 1 \\ 0
    \end{bmatrix}
    &= \mathbf{0}
\end{align*}
% q3
\begin{align*}
 (\mathbf{A}-\lambda_3) \mathbf{q}_3 &= \mathbf{0}\\
 \begin{bmatrix}
     -1 & 0 & -1 \\
     0 & -1 & 0 \\
    0 & 0 & 0
    \end{bmatrix}\mathbf{q}_3 &= \mathbf{0}
\end{align*}

\begin{align*}
  \begin{bmatrix}
      -1 & 0 & -1 \\
     0 & -1 & 0 \\
    0 & 0 & 0
    \end{bmatrix}
    \begin{bmatrix}
     -1 \\ 0 \\1
    \end{bmatrix}
    &= \mathbf{0}
\end{align*}


 Thus the Jordan form is

 \begin{align*}
 \hat{\mathbf{A}}_3 &= \begin{bmatrix}
                        1 & 0 &0\\0 & 1 & 0\\0 & 0&2
                       \end{bmatrix}\\
  \mathbf{Q} &= \begin{bmatrix}
   1 & 0 & -1\\
   0 & 1 & 0\\
   0 & 0 &1
  \end{bmatrix}
 \end{align*}

 %%%%%%%%%%%%%%%%%%%%

 \item $\mathbf{A}_4$

 \begin{align*}
\det(\lambda \mathbf{I} - \mathbf{A}_3) &= \det \left( \begin{bmatrix}
                                   \lambda & -4 & -3\\
                                   0 & \lambda -20 & -16\\
                                   0 & 25 & \lambda +20
                                  \end{bmatrix}
 \right)\\
 &= \lambda((\lambda -20)(\lambda +20)-25(-16))\\
 &= \lambda (\lambda^2 -400+400)\\
 \lambda^3 = 0 &\rightarrow \lambda_{1,2,3} =0
 \end{align*}

 The new representation of A is then

  \begin{equation*}
  \hat{\mathbf{A}}_4 = \mathbf{J}  =  \begin{bmatrix}
                                       \lambda & 1 & 0\\
                                       0 & \lambda & 1\\
                                       0 & 0 & \lambda
                                      \end{bmatrix}
 = \begin{bmatrix}
                       0 & 1 & 0\\ 0&0&1\\0&0&0
                      \end{bmatrix}
 \end{equation*}

 Due to the repeated eigenvalues of the matrix, it is necessary
 to obtain generalized vectors for the matrix.

We obtain the basis.

\begin{align*}
 (\mathbf{A}-\lambda\mathbf{I}) &= \mathbf{A}\\
 (\mathbf{A}-\lambda\mathbf{I})^2 &= \begin{bmatrix}
                                      0 & 5 & 4\\
                                      0 & 0 & 0\\
                                      0 & 0 & 0
                                     \end{bmatrix}
\\
 (\mathbf{A}-\lambda\mathbf{I})^3 &= \begin{bmatrix}
                                      0 & 0 & 0\\
                                      0 & 0 & 0\\
                                      0 & 0 & 0
                                     \end{bmatrix}
\end{align*}

Find a solution to
\begin{align*}
(\mathbf{A}-\lambda \mathbf{I})^3 \mathbf{v}& = \mathbf{0} \\
(\mathbf{A}-\lambda \mathbf{I})^2 \mathbf{v}& = \mathbf{v}_2 \\
(\mathbf{A}-\lambda \mathbf{I}) \mathbf{v}& = \mathbf{v}_3 \\
\end{align*}

\begin{align*}
 (\mathbf{A}-\lambda \mathbf{I})^3 \mathbf{v}& = \mathbf{0} \\
 (\mathbf{A}-\lambda \mathbf{I})^3 = \begin{bmatrix}
  0 & 0 & 0\\
  0 & 0 & 0\\
  0 & 0 & 0\\
 \end{bmatrix}
&\rightarrow \mathbf{v}= \begin{bmatrix}
                            0\\1\\0
                           \end{bmatrix}
\end{align*}
% A^2
\begin{align*}
 (\mathbf{A}-\lambda \mathbf{I})^2 \mathbf{v}& = \mathbf{v}_2 \\
 (\mathbf{A}-\lambda \mathbf{I})^2 = \begin{bmatrix}
  0 & 5 & 4\\
  0 & 0 & 0\\
  0 & 0 & 0\\
 \end{bmatrix}
&\rightarrow \mathbf{v}_2= \begin{bmatrix}
                            5\\0\\0
                           \end{bmatrix}
\end{align*}
% A
\begin{align*}
 (\mathbf{A}-\lambda \mathbf{I}) \mathbf{v}& = \mathbf{v}_3 \\
 (\mathbf{A}-\lambda \mathbf{I}) = \begin{bmatrix}
  0 & 4 & 3\\
  0 & 20 &16\\
  0 & -25 & -20\\
 \end{bmatrix}
&\rightarrow \mathbf{v}_3= \begin{bmatrix}
                            4\\20\\-25
                           \end{bmatrix}
\end{align*}

 Thus the Jordan form is

 \begin{align*}
 \hat {\mathbf{A}}_4 &= \begin{bmatrix}
                       0 & 1 & 0\\ 0&0&1\\0&0&0
                      \end{bmatrix}\\
  \mathbf{Q} &= \begin{bmatrix}
   5 & 4 & 0\\
   0 & 20 & 1\\
   0 & -25 &0
  \end{bmatrix}
 \end{align*}

\end{itemize}
