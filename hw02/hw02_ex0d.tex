\item Find the Jordan-canonical-form
representations of the following matrices:

\begin{align*}
 \mathbf{A}_1 = \begin{bmatrix}
        1 & 4 &10\\
        0 & 2 & 0\\
        0 & 0 & 3
       \end{bmatrix}
& \quad
\mathbf{A}_2 = \begin{bmatrix}
        0 & 1 & 0\\
        0 & 0 & 1\\
        -2 & -4 & -3
       \end{bmatrix}
\\
\mathbf{A}_3 = \begin{bmatrix}
        0 & 4 & 3\\
        0 & -150 & -120\\
        0 & 200 & 160
       \end{bmatrix}
& \quad
\mathbf{A}_4 = \begin{bmatrix}
        0 & 4 & 3\\
        0 & 20 & 16\\
        0 & -25 & -20
       \end{bmatrix}
\\
\mathbf{A}_5 = \begin{bmatrix}
        7/2 & 21/2 & 14\\
        -1/2 & -3/2 & -2\\
        -1/2 & -3/2 &-2
       \end{bmatrix}
& \quad
\mathbf{A}_6 = \begin{bmatrix}
        0 & 1 & 1 & 1 & 1\\
        0 & 0 & 1 & 1 & 1\\
        0 & 0 & 0 & 1 & 1\\
        0 & 0 & 0 & 0 & 1\\
        0 & 0 & 0 & 0 & 0
       \end{bmatrix}
\end{align*}


\begin{itemize}
 \item $\mathbf{A}_1$

 From Problem 3.13 we have:

The matrix is triangular, thus its eigenvalues are the values
 of the diagonal.
 \begin{equation*}
  \hat{\mathbf{A}}_1 = \begin{bmatrix}
                        1 & 0 &0\\0 & 2 & 0\\0 & 0&3
                       \end{bmatrix}
 \end{equation*}

We obtain $\mathbf{Q} = \begin{bmatrix}
                         \mathbf{q}_1 &\mathbf{q}_2 &\mathbf{q}_3
                        \end{bmatrix}
$.
% q1
\begin{align*}
 (\mathbf{A}-\lambda_1) \mathbf{q}_1 &= \mathbf{0}\\
 \begin{bmatrix}
    0 & 4 & 10 \\
     0 & 1 & 0 \\
    0 & 0 & 2
    \end{bmatrix}\mathbf{q}_1 &= \mathbf{0}
\end{align*}

\begin{align*}
  \begin{bmatrix}
    0 & 4 & 10 \\
     0 & 1 & 0 \\
    0 & 0 & 2
    \end{bmatrix}
    \begin{bmatrix}
     1 \\ 0 \\ 0
    \end{bmatrix}
    &= \mathbf{0}
\end{align*}
% q2
\begin{align*}
 (\mathbf{A}-\lambda_2) \mathbf{q}_2 &= \mathbf{0}\\
 \begin{bmatrix}
    -1 & 4 & 10 \\
     0 & 0 & 0 \\
    0 & 0 & 1
    \end{bmatrix}\mathbf{q}_2 &= \mathbf{0}
\end{align*}

\begin{align*}
  \begin{bmatrix}
    -1 & 4 & 10 \\
     0 & 0 & 0 \\
    0 & 0 & 1
    \end{bmatrix}
    \begin{bmatrix}
     4 \\ 1 \\ 0
    \end{bmatrix}
    &= \mathbf{0}
\end{align*}
% q3
\begin{align*}
 (\mathbf{A}-\lambda_3) \mathbf{q}_3 &= \mathbf{0}\\
 \begin{bmatrix}
    -2 & 4 & 10 \\
     0 & -1 & 0 \\
    0 & 0 & 0
    \end{bmatrix}\mathbf{q}_3 &= \mathbf{0}
\end{align*}

\begin{align*}
  \begin{bmatrix}
    -2 & 4 & 10 \\
     0 & -1 & 0 \\
    0 & 0 & 0
    \end{bmatrix}
    \begin{bmatrix}
     5 \\ 0 \\ 1
    \end{bmatrix}
    &= \mathbf{0}
\end{align*}

Thus the Jordan form is

 \begin{align*}
 \hat{\mathbf{A}}_1 &= \begin{bmatrix}
                        1 & 0 &0\\0 & 2 & 0\\0 & 0&3
                       \end{bmatrix}\\
  \mathbf{Q} &= \begin{bmatrix}
   1 & 4 & 5\\
   0 & 1 & 0\\
   0 & 0 &1
  \end{bmatrix}
 \end{align*}

 %%%%%%%%%%%%%%%%%%%%%%%%%%%%%%%%%%%

 \item $\mathbf{A}_2$

 We obtain the eigenvalues of the matrix.

 \begin{align*}
  \det(\lambda \mathbf{I} - \mathbf{A}_2) &=
  \det(\begin{bmatrix}
       \lambda & -1 & 0 \\
       0 & \lambda & -1 \\
       2 & 4 & \lambda+3
      \end{bmatrix})=\mathbf{0}\\
      &= \lambda (\lambda^2 +3\lambda +4) -2\\
      &= \lambda^3 +3\lambda^2 + 4 \lambda -2\\
      &= (\lambda +1) (\lambda +1 + j) (\lambda +1 - j)
 \end{align*}


  \begin{equation*}
  \hat{\mathbf{A}}_2 = \begin{bmatrix}
                        -1 & 0 &0\\0 & -1-j & 0\\0 & 0&-1+j
                       \end{bmatrix}
 \end{equation*}

 Determine the eigenvectors of matrix $\mathbf{A}_2$.

\begin{align*}
 (\mathbf{A}-\lambda_1) \mathbf{q}_1 &= \mathbf{0}\\
 \begin{bmatrix}
    1 & 1 & 0 \\
     0 & 1 & 1 \\
    -2 & -4 & -2
    \end{bmatrix}\mathbf{q}_1 &= \mathbf{0}
\end{align*}

\begin{align*}
  \begin{bmatrix}
     1 & 1 & 0 \\
     0 & 1 & 1 \\
    -2 & -4 & -2
    \end{bmatrix}
    \begin{bmatrix}
     1 \\ -1 \\ 1
    \end{bmatrix}
    &= \mathbf{0}
\end{align*}
% q2
\begin{align*}
 (\mathbf{A}-\lambda_2) \mathbf{q}_2 &= \mathbf{0}\\
 \begin{bmatrix}
     1+j & 1 & 0 \\
     0 & 1+j & 1 \\
    -2 & -4 & -2+j
    \end{bmatrix}\mathbf{q}_2 &= \mathbf{0}
\end{align*}

\begin{align*}
  \begin{bmatrix}
     1+j & 1 & 0 \\
     0 & 1+j & 1 \\
    -2 & -4 & -2+j
    \end{bmatrix}
    \begin{bmatrix}
     -1 \\ 1+j \\ -2j
    \end{bmatrix}
    &= \mathbf{0}
\end{align*}
% q3
\begin{align*}
 (\mathbf{A}-\lambda_3) \mathbf{q}_3 &= \mathbf{0}\\
 \begin{bmatrix}
     1-j & 1 & 0 \\
     0 & 1-j & 1 \\
    -2 & -4 & -2-j
    \end{bmatrix}\mathbf{q}_3 &= \mathbf{0}
\end{align*}

\begin{align*}
  \begin{bmatrix}
     1-j & 1 & 0 \\
     0 & 1-j & 1 \\
    -2 & -4 & -2-j
    \end{bmatrix}
    \begin{bmatrix}
     -1 \\ 1-j \\ 2j
    \end{bmatrix}
    &= \mathbf{0}
\end{align*}

Thus the Jordan form is
 \begin{align*}
 \hat{\mathbf{A}}_2 &= \begin{bmatrix}
                        -1 & 0 &0\\0 & -1-j & 0\\0 & 0&-1+j
                       \end{bmatrix}\\
  \mathbf{Q} &= \begin{bmatrix}
   1 & -1 & -1\\
   -1 & 1+j & 1-j\\
   1 & -2j &2j
  \end{bmatrix}
 \end{align*}

 %%%%%%%%%%%%%%%%%%%%%%%%%%%%%%%%%%%%%

 \item $\mathbf{A}_3$

  \begin{align*}
\det(\lambda \mathbf{I} - \mathbf{A}_3) &= \det \left( \begin{bmatrix}
                                   \lambda & -4 & -3\\
                                   0 & \lambda +150 & 120\\
                                   0 & -200 & \lambda -160
                                  \end{bmatrix}
 \right)\\
 &= \lambda((\lambda +150)(\lambda -160)+200(120))\\
 &= \lambda^2(\lambda-10 )\\
 \lambda^2(\lambda-10 ) = 0 &\rightarrow \lambda_{1,2} =0, \lambda_3 = 10
 \end{align*}

 Due to the repeated roots, it is necessary to
construct generalized eignevectors to form $\mathbf{S}$.

We obtain the basis.

\begin{align*}
 (\mathbf{A}-\lambda\mathbf{I}) &= \begin{bmatrix}
                                    -\lambda & 4 & 3\\
                                    0 &  -150 - \lambda & -120\\
                                    0 & 200 & 160 -\lambda
                                   \end{bmatrix}
\\
 \lambda = 10 & \rightarrow (\mathbf{A}-\lambda\mathbf{I}) = \begin{bmatrix}
                                    -10& 4 & 3\\
                                    0 &  -160 & -120\\
                                    0 & 200 & 150
                                   \end{bmatrix}
\\
 \lambda = 0 & \rightarrow (\mathbf{A}-\lambda\mathbf{I}) = \begin{bmatrix}
                                      0 & 4 & 3\\
                                    0 &  -150 & -120\\
                                    0 & 200 & 160
                                     \end{bmatrix}\\
& \rightarrow (\mathbf{A}-\lambda\mathbf{I})^2 = \begin{bmatrix}
                                      0 & 0 & 0\\
                                      0 & -1500 &-1200\\
                                      0 & 2000 & 1600
                                     \end{bmatrix}
\end{align*}

Find a solution to
\begin{align*}
(\mathbf{A}- 10\mathbf{I}) \mathbf{v}_3& = \mathbf{0} \\
(\mathbf{A}-0 \mathbf{I})^2 \mathbf{v}_2 & = \mathbf{0} \\
(\mathbf{A}- 0\mathbf{I}) \mathbf{v}_2& = \mathbf{v}_1 \\
\end{align*}

\begin{align*}
 (\mathbf{A}- 10\mathbf{I}) \mathbf{v}_3& = \mathbf{0} \\
 (\mathbf{A}-\lambda \mathbf{I}) = \begin{bmatrix}
                                    -10& 4 & 3\\
                                    0 &  -160 & -120\\
                                    0 & 200 & 150
 \end{bmatrix}
&\rightarrow \mathbf{v}_3= \begin{bmatrix}
                            0\\3\\-4
                           \end{bmatrix}
\end{align*}
% A^2
\begin{align*}
 (\mathbf{A}-0 \mathbf{I})^2 \mathbf{v}_2 & = \mathbf{0} \\
 (\mathbf{A}-\lambda \mathbf{I})^2 = \begin{bmatrix}
                                        0 & 0 & 0\\
                                      0 & -1500 &-1200\\
                                      0 & 2000 & 1600
 \end{bmatrix}
&\rightarrow \mathbf{v}_2= \begin{bmatrix}
                            1\\-4\\5
                           \end{bmatrix}
\end{align*}
% A
\begin{align*}
 (\mathbf{A}- 0\mathbf{I}) \mathbf{v}_2& = \mathbf{v}_1 \\
 (\mathbf{A}-\lambda \mathbf{I}) = \begin{bmatrix}
                                    0 & 4 & 3\\
                                    0 &  -150 & -120\\
                                    0 & 200 & 160
 \end{bmatrix}
&\rightarrow \mathbf{v}_1= \begin{bmatrix}
                            -1\\0\\0
                           \end{bmatrix}
\end{align*}

 Thus the Jordan form is

 \begin{align*}
 \mathbf{J} &= \begin{bmatrix}
                       0 & 1 & 0\\ 0&0&0\\0&0&10
                      \end{bmatrix}\\
  \mathbf{S} &= \begin{bmatrix}
   -1 & 1 & 0\\
   0 & -4 & 3\\
   0 & 5 & -4
  \end{bmatrix}
 \end{align*}



 %%%%%%%%%%%%%%%%%%%%%%%%%%%%%%%%%%%%%%

 \item $\mathbf{A}_4$


 \begin{align*}
\det(\lambda \mathbf{I} - \mathbf{A}_3) &= \det \left( \begin{bmatrix}
                                   \lambda & -4 & -3\\
                                   0 & \lambda -20 & -16\\
                                   0 & 25 & \lambda +20
                                  \end{bmatrix}
 \right)\\
 &= \lambda((\lambda -20)(\lambda +20)-25(-16))\\
 &= \lambda (\lambda^2 -400+400)\\
 \lambda^3 = 0 &\rightarrow \lambda_{1,2,3} =0
 \end{align*}

 The new representation of A is then

  \begin{equation*}
  \hat{\mathbf{A}}_4 = \mathbf{J}  =  \begin{bmatrix}
                                       \lambda & 1 & 0\\
                                       0 & \lambda & 1\\
                                       0 & 0 & \lambda
                                      \end{bmatrix}
 = \begin{bmatrix}
                       0 & 1 & 0\\ 0&0&1\\0&0&0
                      \end{bmatrix}
 \end{equation*}

 Due to the repeated eigenvalues of the matrix, it is necessary
 to obtain generalized vectors for the matrix.

We obtain the basis.

\begin{align*}
 (\mathbf{A}-\lambda\mathbf{I}) &= \mathbf{A}\\
 (\mathbf{A}-\lambda\mathbf{I})^2 &= \begin{bmatrix}
                                      0 & 5 & 4\\
                                      0 & 0 & 0\\
                                      0 & 0 & 0
                                     \end{bmatrix}
\\
 (\mathbf{A}-\lambda\mathbf{I})^3 &= \begin{bmatrix}
                                      0 & 0 & 0\\
                                      0 & 0 & 0\\
                                      0 & 0 & 0
                                     \end{bmatrix}
\end{align*}

Find a solution to
\begin{align*}
(\mathbf{A}-\lambda \mathbf{I})^3 \mathbf{v}& = \mathbf{0} \\
(\mathbf{A}-\lambda \mathbf{I})^2 \mathbf{v}& = \mathbf{v}_2 \\
(\mathbf{A}-\lambda \mathbf{I}) \mathbf{v}& = \mathbf{v}_3 \\
\end{align*}

\begin{align*}
 (\mathbf{A}-\lambda \mathbf{I})^3 \mathbf{v}& = \mathbf{0} \\
 (\mathbf{A}-\lambda \mathbf{I})^3 = \begin{bmatrix}
  0 & 0 & 0\\
  0 & 0 & 0\\
  0 & 0 & 0\\
 \end{bmatrix}
&\rightarrow \mathbf{v}= \begin{bmatrix}
                            0\\1\\0
                           \end{bmatrix}
\end{align*}
% A^2
\begin{align*}
 (\mathbf{A}-\lambda \mathbf{I})^2 \mathbf{v}& = \mathbf{v}_2 \\
 (\mathbf{A}-\lambda \mathbf{I})^2 = \begin{bmatrix}
  0 & 5 & 4\\
  0 & 0 & 0\\
  0 & 0 & 0\\
 \end{bmatrix}
&\rightarrow \mathbf{v}_2= \begin{bmatrix}
                            5\\0\\0
                           \end{bmatrix}
\end{align*}
% A
\begin{align*}
 (\mathbf{A}-\lambda \mathbf{I}) \mathbf{v}& = \mathbf{v}_3 \\
 (\mathbf{A}-\lambda \mathbf{I}) = \begin{bmatrix}
  0 & 4 & 3\\
  0 & 20 &16\\
  0 & -25 & -20\\
 \end{bmatrix}
&\rightarrow \mathbf{v}_3= \begin{bmatrix}
                            4\\20\\-25
                           \end{bmatrix}
\end{align*}

 Thus the Jordan form is

 \begin{align*}
 \hat {\mathbf{A}}_4 &= \begin{bmatrix}
                       0 & 1 & 0\\ 0&0&1\\0&0&0
                      \end{bmatrix}\\
  \mathbf{Q} &= \begin{bmatrix}
   5 & 4 & 0\\
   0 & 20 & 1\\
   0 & -25 &0
  \end{bmatrix}
 \end{align*}


 %%%%%%%%%%%%%%%%%%%%%%%
 \item $\mathbf{A}_5$

We obtain the eigenvalues of the matrix.

\begin{align*}
 \det(\lambda \mathbf{I} - \mathbf{A}) &=
 \det\left(
  \begin{bmatrix}
    \lambda -7/2 & -21/2 & -14 \\
    1/2 & \lambda +3/2 & 2\\
    1/2 & 3/2 & \lambda +2
    \end{bmatrix}
\right) = \mathbf{0}
\end{align*}

The eigenvalue is zero with multiplicity of three.

We construct generalized vectors for the matrix
employing the same procedure as Problem 3.13.

We obtain the basis.

\begin{align*}
 (\mathbf{A}-\lambda\mathbf{I}) &= \begin{bmatrix}
        7/2 & 21/2 & 14 \\
        -1/2 & -3/2 & -2 \\
        -1/2 & -3/2 & -2
    \end{bmatrix}\\
 (\mathbf{A}-\lambda\mathbf{I})^2 &= \begin{bmatrix}
                                      0 & 0 & 0\\
                                      0 & 0 & 0\\
                                      0 & 0 & 0
                                     \end{bmatrix}
\\
 (\mathbf{A}-\lambda\mathbf{I})^3 &= \begin{bmatrix}
                                      0 & 0 & 0\\
                                      0 & 0 & 0\\
                                      0 & 0 & 0
                                     \end{bmatrix}
\end{align*}

Find a solution to
\begin{align*}
(\mathbf{A}-\lambda \mathbf{I})^2 \mathbf{v}_3& = \mathbf{0} \\
(\mathbf{A}-\lambda \mathbf{I})^2 \mathbf{v}_2& = \mathbf{0} \\
(\mathbf{A}-\lambda \mathbf{I}) \mathbf{v}_2& = \mathbf{v}_1 \\
\end{align*}

\begin{align*}
 (\mathbf{A}-\lambda \mathbf{I})^2 \mathbf{v}_3& = \mathbf{0} \\
 (\mathbf{A}-\lambda \mathbf{I})^2 = \begin{bmatrix}
  0 & 0 & 0\\
  0 & 0 & 0\\
  0 & 0 & 0\\
 \end{bmatrix}
&\rightarrow \mathbf{v}= \begin{bmatrix}
                            0\\1\\-3/4
                           \end{bmatrix}
\end{align*}
% A^2
\begin{align*}
 (\mathbf{A}-\lambda \mathbf{I})^2 \mathbf{v}_2& = \mathbf{0} \\
 (\mathbf{A}-\lambda \mathbf{I})^2 = \begin{bmatrix}
  0 & 0 & 0 \\
  0 & 0 & 0\\
  0 & 0 & 0\\
 \end{bmatrix}
&\rightarrow \mathbf{v}_2= \begin{bmatrix}
                            0\\0\\1
                           \end{bmatrix}
\end{align*}
% A
\begin{align*}
 (\mathbf{A}-\lambda \mathbf{I}) \mathbf{v}& = \mathbf{v}_3 \\
 (\mathbf{A}-\lambda \mathbf{I}) = \begin{bmatrix}
  0 & 0 & 0\\
  0 & 0 & 0\\
  0 & 0 & 0\\
 \end{bmatrix}
&\rightarrow \mathbf{v}_3= \begin{bmatrix}
                            14\\-2\\-2
                           \end{bmatrix}
\end{align*}

 Thus the Jordan form is

 \begin{align*}
 \hat {\mathbf{A}} &= \begin{bmatrix}
                       0 & 1 & 0 \\
                       0 & 0 & 0 \\
                       0 & 0 & 0
                      \end{bmatrix}\\
  \mathbf{Q} &= \begin{bmatrix}
   14& 0 & 0\\
   -2&0 & 1\\
   -2 & 1 &-3/4
  \end{bmatrix}
 \end{align*}

 %%%%%%%%%%%%%%%%%%%%%%%%%


 \item $\mathbf{A}_6$

 Because the matrix is in triangular form,
 we know that the eigenvalues of the matrix are all zero.


 We obtain the basis for the eigenvectors.

\begin{align*}
 (\mathbf{A}-\lambda\mathbf{I}) = \begin{bmatrix}
        0 & 1 & 1 & 1 & 1\\
        0 & 0 & 1 & 1 & 1\\
        0 & 0 & 0 & 1 & 1\\
        0 & 0 & 0 & 0 & 1\\
        0 & 0 & 0 & 0 & 0
    \end{bmatrix}
    &
 (\mathbf{A}-\lambda\mathbf{I})^2 = \begin{bmatrix}
        0 & 0 & 1 & 0 & 0 \\
        0 & 0 & 0 & 1 & 0 \\
        0 & 0 & 0 & 0 & 1 \\
        0 & 0 & 0 & 0 & 0 \\
        0 & 0 & 0 & 0 & 0
    \end{bmatrix}
\\
 (\mathbf{A}-\lambda\mathbf{I})^3 = \begin{bmatrix}
        0 & 0 & 0 & 1 & 0 \\
        0 & 0 & 0 & 0 & 1 \\
        0 & 0 & 0 & 0 & 0 \\
        0 & 0 & 0 & 0 & 0 \\
        0 & 0 & 0 & 0 & 0
    \end{bmatrix}
&
 (\mathbf{A}-\lambda\mathbf{I})^4 = \begin{bmatrix}
        0 & 0 & 0 & 0 & 1\\
        0 & 0 & 0 & 0 & 0\\
        0 & 0 & 0 & 0 & 0\\
        0 & 0 & 0 & 0 & 0\\
        0 & 0 & 0 & 0 & 0
        \end{bmatrix}
\\
 (\mathbf{A}-\lambda\mathbf{I})^5 = \begin{bmatrix}
        0 & 0 & 0 & 0 & 0\\
        0 & 0 & 0 & 0 & 0\\
        0 & 0 & 0 & 0 & 0\\
        0 & 0 & 0 & 0 & 0\\
        0 & 0 & 0 & 0 & 0
        \end{bmatrix}&
\end{align*}

Find a solution to
\begin{align*}
(\mathbf{A}-\lambda \mathbf{I})^5 \mathbf{v}_5& = \mathbf{0} \\
(\mathbf{A}-\lambda \mathbf{I})^4 \mathbf{v}_5& = \mathbf{v}_4 \\
(\mathbf{A}-\lambda \mathbf{I})^3 \mathbf{v}_4& = \mathbf{v}_3 \\
(\mathbf{A}-\lambda \mathbf{I})^2 \mathbf{v}_3& = \mathbf{v}_2 \\
(\mathbf{A}-\lambda \mathbf{I}) \mathbf{v}_2& = \mathbf{v}_1 \\
\end{align*}


 Thus the Jordan form is

 \begin{align*}
 \hat {\mathbf{A}} &= \begin{bmatrix}
                       0 & 1 & 0 & 0 & 0\\
                       0 & 0 & 1 & 0 & 0\\
                        0 & 0 & 0 & 1 & 0\\
                        0 & 0 & 0 & 0 & 1\\
                        0 & 0 & 0 & 0 & 0\\
                      \end{bmatrix}\\
  \mathbf{Q} &= \begin{bmatrix}
                    1 & 0 & 0 & 0 & 0\\
                    0 & 1 & -1 & 1 & -1\\
                    0 & 0 & 1 & -2 & 3\\
                    0 & 0 & 0 & 1 & -3\\
                    0 & 0 & 0 & 0 & 1
  \end{bmatrix}
 \end{align*}

\end{itemize}
