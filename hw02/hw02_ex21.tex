\item [3.21] Given

\begin{equation*}
 A = \begin{bmatrix}
      1 & 1 & 0\\
      0 & 0 & 1\\
      0 & 0 & 1
     \end{bmatrix}
\end{equation*}

find $\mathbf{A}^{10}$, $\mathbf{A}^{103}$, and $e^{\mathbf{A}t}$.

We first determine the eigenvalues of $\mathbf{A}$.

\begin{equation*}
 \lambda_1 = 1  \qquad \lambda_2 = 0  \qquad  \lambda_3 = 1
\end{equation*}

Let $h(\lambda) = \beta_0 + \beta_1 \lambda + \beta_2 \lambda^2$.
From Cayley-Hamilton's theorem, we know that $f(\lambda) = f(\mathbf{A})$.
Using this property, we calculate the three functions given by finding
linear combinations of the basis $\begin{bmatrix}
                                   \mathbf{I} & \mathbf{A} & \mathbf{A}^2
                                  \end{bmatrix}
$


\begin{itemize}
 \item $\mathbf{A}^{10}$

 Let $f(\lambda) = \lambda^{10}$.
 Then $\frac{d f(\lambda)}{d\lambda} = 10 \lambda^9$.

 \begin{align*}
  \lambda = 0 & \rightarrow f(0) = 0^{10} = \beta_0 = 0\\
  \lambda = 1 & \rightarrow f(1) = 1^{10} = \beta_1 + \beta_2 = 1\\
  &\rightarrow \frac{d f(\lambda)}{d\lambda}\Bigr|_{\substack{\lambda=1}} = 10 = \beta_1 + 2 \beta_2\\
  \beta_0 &= 0\\
  \beta_1 &= -8\\
  \beta_2 &= 9\\
  \mathbf{A}^2 &= \begin{bmatrix}
                   1 & 1&1\\
                   0 &0&1\\
                   0 &0 &1
                  \end{bmatrix}\\
    \mathbf{A}^{10} &= \beta_0 \mathbf{I} +  \beta_1 \mathbf{A} + \beta_2 \mathbf{A}^2\\
    &= -8 \begin{bmatrix}
           1& 1& 0\\
           0 & 0 & 1\\
           0 & 0 &1
          \end{bmatrix}
+
9       \begin{bmatrix}
            1 & 1&1\\
            0 &0&1\\
            0 &0 &1
        \end{bmatrix}\\
         \mathbf{A}^{10} &=  \begin{bmatrix}
            1 & 1&9\\
            0 &0&1\\
            0 &0 &1
        \end{bmatrix}\\
 \end{align*}


 \item $\mathbf{A}^{103}$

  Let $f(\lambda) = \lambda^{103}$.
 Then $\frac{d f(\lambda)}{d\lambda} = 103 \lambda^{102}$.

 \begin{align*}
  \lambda = 0 & \rightarrow f(0) = 0^{103} = \beta_0 = 0\\
  \lambda = 1 & \rightarrow f(1) = 1^{103} = \beta_1 + \beta_2 = 1\\
  &\rightarrow \frac{d f(\lambda)}{d\lambda}\Bigr|_{\substack{\lambda=1}} = 103 = \beta_1 + 2 \beta_2\\
  \beta_0 &= 0\\
  \beta_1 &= -101\\
  \beta_2 &= 102\\
\end{align*}

\begin{align*}
    \mathbf{A}^{103} &= \beta_0 \mathbf{I}+  \beta_1 \mathbf{A} + \beta_2 \mathbf{A}^2\\
    &= -101 \begin{bmatrix}
           1& 1& 0\\
           0 & 0 & 1\\
           0 & 0 &1
          \end{bmatrix}
+
102       \begin{bmatrix}
            1 & 1&1\\
            0 &0&1\\
            0 &0 &1
        \end{bmatrix}\\
         \mathbf{A}^{103} &=  \begin{bmatrix}
            1 & 1&102\\
            0 &0&1\\
            0 &0 &1
        \end{bmatrix}\\
 \end{align*}

 \item $e^{\mathbf{A}t}$
\end{itemize}

  Let $f(\lambda) = \exp(\lambda t)$.
 Then $\frac{d f(\lambda)}{d\lambda} = t \exp(\lambda t)$.

 \begin{align*}
  \lambda = 0 & \rightarrow f(0) = \exp(0) = \beta_0 = 1\\
  \lambda = 1 & \rightarrow f(1) = \exp(t) = 1 + \beta_1 + \beta_2 \\
  &\rightarrow \frac{d f(\lambda)}{d\lambda}\Bigr|_{\substack{\lambda=1}} = t \exp(t) = \beta_1 + 2 \beta_2\\
  \beta_0 &= 1\\
  \beta_1 &= (2-t) \exp(t) -2\\
  \beta_2 &= (t-1) \exp(t) +1\\
\end{align*}

\begin{align*}
    e^{\mathbf{A}t} &= \beta_0 \mathbf{I} +  \beta_1 \mathbf{A} + \beta_2 \mathbf{A}^2\\
    &= \mathbf{I} +  ((2-t) e^t -2) \begin{bmatrix}
           1& 1& 0\\
           0 & 0 & 1\\
           0 & 0 &1
          \end{bmatrix}
+
((t-1) e^t +1)       \begin{bmatrix}
            1 & 1&1\\
            0 &0&1\\
            0 &0 &1
        \end{bmatrix}\\
         e^{\mathbf{A}t} &=  \begin{bmatrix}
            e^t & e^t -1 & t e^t - e^t +1 \\
            0 & 1 & e^t -1\\
            0 & 0 & e^t
        \end{bmatrix}\\
 \end{align*}

