\item[3.24] Let

\begin{equation*}
\mathbf{C} = \begin{bmatrix}
 \lambda_1 & 0 & 0\\
 0 &  \lambda_2 & 0\\
 0 & 0 & \lambda_3
\end{bmatrix}
\end{equation*}

Find a matrix $\mathbf{B}$ such that $e^{\mathbf{B}} = \mathbf{C}$.
Show that if $\lambda_i = 0$ for some i, then $\mathbf{B}$ does not exist.
Let
\begin{equation*}
 \mathbf{C} = \begin{bmatrix}
      \lambda & 1 & 0\\
      0 & \lambda & 0\\
      0 & 0 & \lambda
     \end{bmatrix}
\end{equation*}

Find a $\mathbf{B}$ such that $e^{\mathbf{B}} = \mathbf{C}$.
Is it true that, for any nonsingular $\mathbf{C}$,
there exists a matrix $\mathbf{B}$ such that $e^{\mathbf{B}} = \mathbf{C}$?


Using properties of logarithms we can find $\mathbf{B}$ as a
function of $\mathbf{C}$.

\begin{align*}
 \exp(\mathbf{B})   &= \mathbf{C}\\
                    &= \begin{bmatrix}
                        \lambda_1 & 0 & 0\\
                        0 & \lambda_2 & 0\\
                        0 & 0 & \lambda_3
                       \end{bmatrix}\\
\ln(\exp(\mathbf{B})) &= \ln(\mathbf{C})\\
\mathbf{B} &= \begin{bmatrix}
                \ln(\lambda_1) & 0 & 0\\
                0 & \ln(\lambda_2 )& 0\\
                0 & 0 & \ln(\lambda_3)
                \end{bmatrix}
\end{align*}

This holds true as long as $\lambda_i \neq 0$ because $\ln(0)$ is not defined.

For

\begin{equation*}
 \mathbf{C} = \begin{bmatrix}
      \lambda & 1 & 0\\
      0 & \lambda & 0\\
      0 & 0 & \lambda
     \end{bmatrix}
\end{equation*}

we employ the representation $\mathbf{Q} \hat{\mathbf{C}} \mathbf{Q}^{-1}$.

\begin{align*}
 \exp(\mathbf{B})   &= \mathbf{C}\\
                    &= \mathbf{Q} \hat{\mathbf{C}} \mathbf{Q}^{-1}\\
                    &= \begin{bmatrix}
                        \lambda & 1 & 0\\
                        0 & \lambda & 0\\
                        0 & 0 & \lambda
                       \end{bmatrix}\\
\ln(\exp(\mathbf{B})) &= \ln(\mathbf{C})\\
\mathbf{B} &= \begin{bmatrix}
                \ln(\lambda) & 1/\lambda & 0\\
                0 & \ln(\lambda ) & 0\\
                0 & 0 & \ln(\lambda)
                \end{bmatrix}\\
            &= \mathbf{Q} \ln(\hat{\mathbf{C}}) \mathbf{Q}^{-1}\\
\end{align*}

For $\hat{\mathbf{C}}$ to exist, $\mathbf{C}$ must be nonsingular.
