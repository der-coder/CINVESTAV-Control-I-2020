\item [3.16] Show that the companion-form matrix in Problem 3.14 is nonsingular if and only if $\alpha_4 \neq 0$.
Under this assumption, show that its inverse equals

\begin{equation*}
 A^{-1} = \begin{bmatrix}
           0 & 1 & 0 & 0 \\
           0 & 0 & 1 & 0 \\
           0 & 0 & 0 & 1 \\
           -1/\alpha_4 & -\alpha_1/\alpha_4 & -\alpha_2 /\alpha_4 & -\alpha_3 /\alpha_4 \\
          \end{bmatrix}
\end{equation*}


We test the assumption by multiplying $\mathbf{A}$ with the proposed inverse.

\begin{align*}
 \mathbf{A}\mathbf{A}^{-1} &=
        \begin{bmatrix}
      -\alpha_1 & -\alpha_2 & -\alpha_3 & -\alpha_4\\
      1 & 0 & 0 & 0\\
      0 & 1 & 0 & 0\\
      0 & 0 & 1 & 0
     \end{bmatrix}
        \begin{bmatrix}
           0 & 1 & 0 & 0 \\
           0 & 0 & 1 & 0 \\
           0 & 0 & 0 & 1 \\
           -1/\alpha_4 & -\alpha_1/\alpha_4 & -\alpha_2 /\alpha_4 & -\alpha_3 /\alpha_4 \\
          \end{bmatrix}\\
    &= \begin{bmatrix}
        -\alpha_4(-1/\alpha_4) &
        -\alpha_1 -\alpha_4(-\alpha_1/\alpha_4) &
        -\alpha_2 -\alpha_4(-\alpha_2/\alpha_4) &
        -\alpha_3 -\alpha_4(-\alpha_3/\alpha_4) \\
        0 & 1 & 0 & 0\\
        0 & 0 & 1 & 0\\
        0 & 0 & 0 & 1
       \end{bmatrix}\\
    &= \begin{bmatrix}
        1 & 0 & 0 & 0\\
        0 & 1 & 0 & 0\\
        0 & 0 & 1 & 0\\
        0 & 0 & 0 & 1
       \end{bmatrix}\\
       &= \mathbf{I}
\end{align*}

The proposed matrix is indeed the inverse of $\mathbf{A}$.
For this to be true, $\alpha_4$  must be different from zero.
If not, then the matrix becomes singular.
