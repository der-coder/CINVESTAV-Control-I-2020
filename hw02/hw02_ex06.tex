\item [3.6] Find bases of the range spaces and null spaces
of the matrices in Problem 3.5.

\begin{itemize}
 \item $\mathbf{A}_1$

 Given that $\mathbf{A}_1 = [\mathbf{a}_1 \,
\mathbf{a}_2 \, \mathbf{a}_3 ]$, we can construct a basis
for the range space of the matrix with columns
$\mathbf{a}_2$ and $\mathbf{a}_3$.

\begin{equation*}
\text{range space}(\mathbf{A}_1) =
\left \{ \mathbf{a}_2 , \mathbf{a}_3 \right \} =
\left \{ \begin{bmatrix} 1 \\ 0 \\ 0  \end{bmatrix},
\begin{bmatrix} 0\\ 0\\ 1 \end{bmatrix}
\right \}
\end{equation*}

The basis for the null space of $\mathbf{A}_1$ is determined
by finding a vector $\mathbf{n}$ that satisfies the equation

\begin{equation*} \mathbf{A}_1 \mathbf{n} = \mathbf{0}
\end{equation*}

Let $\mathbf{n} = \begin{bmatrix} n_{11} & n_{21} & n_{31}
\end{bmatrix}'$ and solve the previous system of equations
for $\mathbf{n}$.

\begin{align*}
    \mathbf{A}_1 \mathbf{n} &=    \begin{bmatrix}
                                        0 & 1 & 0\\
                                        0 & 0 & 0\\
                                        0 & 0 & 1
                                    \end{bmatrix}
                                    \begin{bmatrix}
                                    n_{11} \\ n_{21} \\ n_{31}
                                    \end{bmatrix}
                                = \mathbf{0}\\
                            &= \begin{bmatrix}
                                    0 + n_{21} + 0\\
                                    0 + 0 + 0\\
                                    0 + 0 + n_{31}
                                \end{bmatrix}
                            = \begin{bmatrix}
                                    0\\0\\0
                                \end{bmatrix} \\
                    n_{21} = 0 & \qquad  n_{31} = 0\\
                    \mathbf{n} &= \begin{bmatrix} n_{11} \\0
\\0 \end{bmatrix}
\end{align*}

We propose $n_{11} = 1$ as a particular solution.

\begin{equation*}
\text{null space}(\mathbf{A}_1) =
\left \{\mathbf{n} \right\} =
\left \{ \begin{bmatrix} 1 \\ 0 \\0 \end{bmatrix} \right \}
\end{equation*}

 \item $\mathbf{A}_2$

 Matrix $\mathbf{A}_2$ is full rank,
 thus its columns can form a basis for the range space.
 By consequence the only solution to $\mathbf{A}_2 \mathbf{n} = \mathbf{0}$
 is the trivial solution.

 \begin{align*}
 \text{range space}(\mathbf{A}_2)
    &=
    \left \{
    \begin{bmatrix} 4 \\ 3 \\ 1 \end{bmatrix},
    \begin{bmatrix} 1\\ 2 \\ 1  \end{bmatrix},
    \begin{bmatrix} -1\\  0\\ 0 \end{bmatrix}
    \right \}
    \\
    \text{null space}(\mathbf{A}_2) &= \begin{bmatrix}
                                        0\\0\\0
                                       \end{bmatrix}
\end{align*}

 \item $\mathbf{A}_3$

 For matrix $\mathbf{A}_3$, three of its four columns can be
used as a basis for the range space. It is only necessary
for the set of vectors to be linearly independent. As such,
the basis of $\mathbf{A}_3$ is constructed as follows

\begin{align*}
\mathbf{A}_3 &= \begin{bmatrix}
                    1 & 2 & 3 & 4\\
                    0 & -1 & -2 & 2\\
                    0 & 0 & 0 & 1
                \end{bmatrix}
            =   \begin{bmatrix}
                    \mathbf{a}_1 &
                    \mathbf{a}_2 &
                    \mathbf{a}_3 &
                    \mathbf{a}_4
                \end{bmatrix} \\
\text{range space}(\mathbf{A}_3) &=
 \left \{ \mathbf{a}_1 , \mathbf{a}_2 , \mathbf{a}_4 \right \}
    =
    \left \{
    \begin{bmatrix} 1 \\  0  \\ 0  \end{bmatrix},
    \begin{bmatrix} 2\\   -1\\ 0  \end{bmatrix},
    \begin{bmatrix}  4\\  2 \\  1 \end{bmatrix}
    \right \}
\end{align*}


As for the basis for the null space of $\mathbf{A}_3$,
it is necessary to solve the equation
$\mathbf{A}_3 \mathbf{n} = \mathbf{0}$.

\begin{align*}
\mathbf{A}_3 \mathbf{n} =    \begin{bmatrix}
                                        1 & 2 & 3 & 4\\
                                        0 & -1 & -2 & 2\\
                                        0 & 0 & 0 & 1
                                    \end{bmatrix}
                                    \begin{bmatrix}
                                    n_{11} \\ n_{21} \\ n_{31} \\ n_{41}
                                    \end{bmatrix}
                                    &= \begin{bmatrix}
        n_{11} + 2 n_{21} + 3 n_{31} + 4 n_{41}\\
        0 - n_{21} - 2 n_{31} + 2 n_{41}\\
        0 + 0 + 0 +  n_{41}
       \end{bmatrix}\\
    &= \begin{bmatrix}
       0 \\ 0 \\ 0
      \end{bmatrix}\\
n_{41} &= 0\\
- n_{21} - 2 n_{31} + 2 (0) = 0
    & \rightarrow n_{21} = - 2 n_{31} \\
n_{11} + 2 (-2n_{31}) + 3 n_{31} + 4 (0) = 0
    & \rightarrow n_{11} = n_{31}\\
     \mathbf{n} &= \begin{bmatrix}
                 n_{11}\\
                 -2 n_{11}\\
                 n_{11}\\
                 0
                \end{bmatrix}
\end{align*}

We propose $n_{11} = 1$ as a particular solution.

\begin{equation*}
  \text{null space}(\mathbf{A}_3) = \mathbf{n} = \begin{bmatrix}
                 1\\
                 -2\\
                 1\\
                 0
                \end{bmatrix}
\end{equation*}




\end{itemize}



