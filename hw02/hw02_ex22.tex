\item [3.22] Use two different methods to compute $e^{\mathbf{A}t}$ for
$\mathbf{A}_1$ and $\mathbf{A}_4$ in Problem 3.13.

\begin{itemize}
 \item $\mathbf{A}_1$

 %% Method 1
The eigenvalues of the matrix are $\lambda_1 = 1, \lambda_2 = 2, \lambda_3 = 3$.
 Let $f(\lambda) = \exp(\lambda t)$.

 \begin{align*}
  \lambda = 1 & \rightarrow f(1) = \exp(t) = \beta_0 + \beta_1 + \beta_2\\
  \lambda = 2 & \rightarrow f(2) = \exp(2t) = \beta_0 + 2 \beta_1 + 4\beta_2 \\
  \lambda = 3 & \rightarrow f(3) = \exp(3t) = \beta_0 + 3\beta_1 + 9\beta_2 \\
%
  \beta_0 &= 3\exp(t) + 3 \exp(2t) + \exp(3t)\\
  \beta_1 &= 4 \exp(2t) -1.5 \exp(3t) -2.5 \exp(t)\\
  \beta_2 &= 0.5(\exp(3t) -2\exp(2t) + \exp(t))
\end{align*}

\begin{align*}
    e^{\mathbf{A}t} &= \beta_0 \mathbf{I} +  \beta_1 \mathbf{A} + \beta_2 \mathbf{A}^2\\
    &= \beta_0 \mathbf{I} +   \beta_1 \begin{bmatrix}
           1& 4& 10\\
           0 & 2 & 1\\
           0 & 0 &3
          \end{bmatrix}
+
 \beta_2       \begin{bmatrix}
            1 & 12 & 40\\
            0 & 4 & 0\\
            0 & 0 & 9
        \end{bmatrix}\\
         e^{\mathbf{A}t} &=  \begin{bmatrix}
            e^t & 4 (e^{2t} - e^t) & 5 (e^{3t} - e^t) \\
            0 & e^{2t} & 0\\
            0 & 0 & e^{3t}
        \end{bmatrix}
 \end{align*}

 %% Method 2

 Employing the representation $\mathbf{A} = \mathbf{Q} \hat{\mathbf{A}} \mathbf{Q}^{-1} $.

 \begin{align*}
  \mathbf{A}_1 &= \begin{bmatrix}
                   1 & 4 & 5\\
                   0 & 1 & 0\\
                   0 & 0 & 1
                  \end{bmatrix}
                  \begin{bmatrix}
                   1 & 0 & 0\\
                   0 & 2 & 0\\
                   0 & 0 & 3
                  \end{bmatrix}
                  \begin{bmatrix}
                   1 & 4 & 5\\
                   0 & 1 & 0\\
                   0 & 0 & 1
                  \end{bmatrix}^{-1}\\
e^{\mathbf{A}_1 t} &= \begin{bmatrix}
                   1 & 4 & 5\\
                   0 & 1 & 0\\
                   0 & 0 & 1
                  \end{bmatrix}
                  \begin{bmatrix}
                   e^t & 0 & 0\\
                   0 & e^{2t} & 0\\
                   0 & 0 & e^{3t}
                  \end{bmatrix}
                  \begin{bmatrix}
                   1 & 4 & 5\\
                   0 & 1 & 0\\
                   0 & 0 & 1
                  \end{bmatrix}^{-1}\\
                &= \begin{bmatrix}
                    e^t & 4 (e^{2t} - e^t) & 5 (e^{3t} - e^t) \\
                    0 & e^{2t} & 0\\
                    0 & 0 & e^{3t}
                   \end{bmatrix}
 \end{align*}


%%%%%%%%%%%%%%%%%%

 \item $\mathbf{A}_4$

 %% Method1
There is one eigenvector $\lambda = 0$ with duplicity of 3.
 Let $f(\lambda) = \exp(\lambda t)$.
 Then $\frac{d f(\lambda)}{d\lambda} = t \exp(\lambda t)$
 and
 $\frac{d^2 f(\lambda)}{d\lambda^2} = t^2 \exp(\lambda t)$
 .

 \begin{align*}
  \lambda = 0 & \rightarrow f(0) = \exp(0) = \beta_0 = 1\\
    &\rightarrow
    \frac{d f(\lambda)}{d\lambda}\Bigr|_{\substack{\lambda=1}}
    = t \exp(0) = t = \beta_1\\
    &\rightarrow
    \frac{d^2 f(\lambda)}{d\lambda^2}\Bigr|_{\substack{\lambda=1}}
    = t^2 \exp(0)
    = t^2
    = 2 \beta_2\\
  \beta_0 &= 1\\
  \beta_1 &= t\\
  \beta_2 &= \frac{t^2}{2}\\
\end{align*}

\begin{align*}
    e^{\mathbf{A}t} &= \beta_0 \mathbf{I} +  \beta_1 \mathbf{A} + \beta_2 \mathbf{A}^2\\
    &= \mathbf{I} +  t \begin{bmatrix}
           0 & 4 & 3\\
           0 & 20 & 16\\
           0 & -25 & -20
          \end{bmatrix}
+
\frac{t^2}{2}  \begin{bmatrix}
            0 & 5 & 4\\
            0 & 0 & 0\\
            0 & 0 & 0
        \end{bmatrix}\\
         e^{\mathbf{A}t} &=  \begin{bmatrix}
            1 & 4t +2.5 t^2 & 3t +2t^2\\
            0 & 1+20t & 16t \\
            0 & -25 t & 1-20t
        \end{bmatrix}\\
 \end{align*}

 %% Method 2
  Employing the representation $\mathbf{A} = \mathbf{Q} \hat{\mathbf{A}} \mathbf{Q}^{-1} $.

 \begin{align*}
  \mathbf{A}_4 &= \begin{bmatrix}
                   5 & 4 & 0\\
                   0 & 20 & 1\\
                   0 & -25 & 0
                  \end{bmatrix}
                  \begin{bmatrix}
                   0 & 1 & 0\\
                   0 & 0 & 1\\
                   0 & 0 & 0
                  \end{bmatrix}
                  \begin{bmatrix}
                   5 & 4 & 0\\
                   0 & 20 & 1\\
                   0 & -25 & 0
                  \end{bmatrix}^{-1}\\
e^{\mathbf{A}_4 t} &= \begin{bmatrix}
                   5 & 4 & 0\\
                   0 & 20 & 1\\
                   0 & -25 & 0
                  \end{bmatrix}
                  \begin{bmatrix}
                   1 & t & t^2/2\\
                   0 & 1 & t\\
                   0 & 0 & 1
                  \end{bmatrix}
                  \begin{bmatrix}
                   5 & 4 & 0\\
                   0 & 20 & 1\\
                   0 & -25 & 0
                  \end{bmatrix}^{-1}\\
                &= \begin{bmatrix}
                    1 & 4t +2.5 t^2 & 3t +2t^2\\
                    0 & 1+20t & 16t \\
                    0 & -25 t & 1-20t
                   \end{bmatrix}
 \end{align*}
\end{itemize}
