\item[3.17] Consider
\begin{equation*}
 \mathbf{A} = \begin{bmatrix}
      \lambda & \lambda T & \lambda T^2 / 2 \\
      0 &\lambda & \lambda T \\
      0 & 0 & \lambda
     \end{bmatrix}
\end{equation*}

with $\lambda \neq 0$ and $T > 0$.
Show that $[0 \; 0 \; 1]'$ is a generalized eigenvector of grade 3 and the three columns of
\begin{equation*}
 \mathbf{Q} = \begin{bmatrix}
      \lambda^2 T^2 & \lambda T^2 & 0\\
      0 & \lambda T & 0\\
      0 & 0 & 1
     \end{bmatrix}
\end{equation*}
 constitute a chain of generalized eigenvectors of length 3.
 Verify

 \begin{equation*}
  \mathbf{Q}^{-1} \mathbf{A} \mathbf{Q} = \begin{bmatrix}
                \lambda & 1 & 0\\
                0 & \lambda & 1\\
                0 & 0 & \lambda
               \end{bmatrix}
 \end{equation*}

 Obtain the eigenvectors of $\mathbf{A}$.

 \begin{equation*}
  (\mathbf{A} - \lambda \mathbf{I})
  = \begin{bmatrix}
        0 & \lambda T & \lambda T^2 / 2 \\
        0 & 0 & \lambda T \\
        0 & 0 & 0
    \end{bmatrix}
 \end{equation*}

Given $T > 0$ and $\lambda \neq 0$ then $\rho (\mathbf{A}) = 2$.
Thus its nullity is one.

We construct generalized vectors for the matrix
employing the same procedure as Problem 3.13.

We obtain the basis.

\begin{align*}
 (\mathbf{A}-\lambda\mathbf{I}) &= \begin{bmatrix}
        0 & \lambda T & \lambda T^2 / 2 \\
        0 & 0 & \lambda T \\
        0 & 0 & 0
    \end{bmatrix}\\
 (\mathbf{A}-\lambda\mathbf{I})^2 &= \begin{bmatrix}
                                      0 & 0 & \lambda^2 T^2\\
                                      0 & 0 & 0\\
                                      0 & 0 & 0
                                     \end{bmatrix}
\\
 (\mathbf{A}-\lambda\mathbf{I})^3 &= \begin{bmatrix}
                                      0 & 0 & 0\\
                                      0 & 0 & 0\\
                                      0 & 0 & 0
                                     \end{bmatrix}
\end{align*}

Find a solution to
\begin{align*}
(\mathbf{A}-\lambda \mathbf{I})^3 \mathbf{v}& = \mathbf{0} \\
(\mathbf{A}-\lambda \mathbf{I})^2 \mathbf{v}& = \mathbf{v}_2 \\
(\mathbf{A}-\lambda \mathbf{I}) \mathbf{v}& = \mathbf{v}_3 \\
\end{align*}

\begin{align*}
 (\mathbf{A}-\lambda \mathbf{I})^3 \mathbf{v}& = \mathbf{0} \\
 (\mathbf{A}-\lambda \mathbf{I})^3 = \begin{bmatrix}
  0 & 0 & 0\\
  0 & 0 & 0\\
  0 & 0 & 0\\
 \end{bmatrix}
&\rightarrow \mathbf{v}= \begin{bmatrix}
                            0\\0\\1
                           \end{bmatrix}
\end{align*}
% A^2
\begin{align*}
 (\mathbf{A}-\lambda \mathbf{I})^2 \mathbf{v}& = \mathbf{v}_2 \\
 (\mathbf{A}-\lambda \mathbf{I})^2 = \begin{bmatrix}
  0 & 0 & \lambda^2 T^2 \\
  0 & 0 & 0\\
  0 & 0 & 0\\
 \end{bmatrix}
&\rightarrow \mathbf{v}_2= \begin{bmatrix}
                            \lambda^2 T^2\\0\\0
                           \end{bmatrix}
\end{align*}
% A
\begin{align*}
 (\mathbf{A}-\lambda \mathbf{I}) \mathbf{v}& = \mathbf{v}_3 \\
 (\mathbf{A}-\lambda \mathbf{I}) = \begin{bmatrix}
  0 & \lambda T & \lambda T^2/2\\
  0 & 0 & \lambda T\\
  0 & 0 & 0\\
 \end{bmatrix}
&\rightarrow \mathbf{v}_3= \begin{bmatrix}
                            \lambda T^2 / 2 \\ \lambda T \\ 0
                           \end{bmatrix}
\end{align*}

 Thus the Jordan form is

 \begin{align*}
 \hat {\mathbf{A}} &= \begin{bmatrix}
                       \lambda & 1 & 0 \\
                       0 & \lambda & 1 \\
                       0 & 0 & \lambda
                      \end{bmatrix}\\
  \mathbf{Q} &= \begin{bmatrix}
   \lambda^2 T^2 & \lambda T^2 /2& 0\\
      0 & \lambda T & 0\\
      0 & 0 & 1
  \end{bmatrix}
 \end{align*}

 We verify $ \hat{\mathbf{A}}  = \mathbf{Q}^{-1} \mathbf{A} \mathbf{Q} $
 by performing the multiplication of
 $\mathbf{Q} \hat{\mathbf{A}} =  \mathbf{A} \mathbf{Q} $.

 \begin{align*}
  \mathbf{A} \mathbf{Q}  &= \begin{bmatrix}
      \lambda & \lambda T & \lambda T^2 / 2 \\
      0 &\lambda & \lambda T \\
      0 & 0 & \lambda
     \end{bmatrix}
     \begin{bmatrix}
   \lambda^2 T^2 & \lambda T^2 & 0\\
      0 & \lambda T & 0\\
      0 & 0 & 1
  \end{bmatrix}\\
  &= \begin{bmatrix}
      \lambda^3 & 3\lambda^2T^2/2 & \lambda T^2/2\\
      0 & \lambda^2 T & 0\\
      0 & 0 & \lambda
     \end{bmatrix}\\
  %%%%%%%%%%%%%%%%%%%%
  \mathbf{Q} \hat{\mathbf{A}} &= \begin{bmatrix}
   \lambda^2 T^2 & \lambda T^2 & 0\\
      0 & \lambda T & 0\\
      0 & 0 & 1
  \end{bmatrix}
  \begin{bmatrix}
      \lambda & \lambda T & \lambda T^2 / 2 \\
      0 &\lambda & \lambda T \\
      0 & 0 & \lambda
     \end{bmatrix}\\
     &= \begin{bmatrix}
      \lambda^3 & 3\lambda^2T^2/2 & \lambda T^2/2\\
      0 & \lambda^2 T & 0\\
      0 & 0 & \lambda
     \end{bmatrix}
 \end{align*}

 It is the same matrix, thus we have proofed the original statement.

 \textbf{Note} The matrix $\mathbf{Q}$ presented in the intructions
 is wrong.
 The answer provided meets the criteria, whilst the original one does not.
